\documentclass{article}
\usepackage[utf8]{inputenc}
\usepackage[spanish]{babel}
\usepackage{listings}
\usepackage{graphicx}
\graphicspath{ {images/} }
\usepackage{cite}

\begin{document}

\begin{titlepage}
    \begin{center}
        \vspace*{1cm}
            
        \Huge
        \textbf{Proyecto final}
            
        \vspace{0.5cm}
        \LARGE
        Primeros Pasos
            
        \vspace{1.5cm}
            
        \textbf{Jesús Antonio Ibarra Agudelo}
            
        \vfill
            
        \vspace{0.8cm}
            
        \Large
        Despartamento de Ingeniería Electrónica y Telecomunicaciones\\
        Universidad de Antioquia\\
        Medellín\\
        Marzo de 2021
            
    \end{center}
\end{titlepage}

\tableofcontents
\newpage
\section{Introducción}\label{intro}
El proyecto final es una parte fundamental del curso ya que nos da la oportunidad de mostrar y aplicar todo lo aprendido. Esta  actividad es importante, corresponde a la ideación del mismo, para esto se debe plasmar todas las ideas que se tengan en un documento, esto nos ayudara a tener una guia para mas adelante, cuando creamos que estamos atascados, podamos recordar lo que queremos hacer y cual es nuestro objetivo. Ademas se debe tener en cuenta que estas ideas de mi compañero, ya que con él, se desarrollara de manera mas facil y de diversos puntos de vista.

\newpage

\section{Contenido} \label{contenido}
Por ahora se hara una lluvia de ideas, que se utilizaran como base para definir el tipo de juego que se quiere hacer. 

\subsection{Ideas Principales}
A partir de este punto seran las ideas principales que se tendran en cuenta a la hora de desarrollar el juego:

\begin{enumerate}
    \item El nombre del juego sera: " Adventure World ", aunque el nombre podra cambiarse mas adelante dependiendo de los avances que se hagan.
    \item El juego pertenecera a la categoria de accion, se tratara de ser lo mas aproximado a un RPG, los jugadores podran escoger entre un luchador de corta distancia o un luchador de larga distancia, aunque estas clases tambien se definiran o cambiaran dependiendo de los avances y contratiempos a la hora de diseñar el juego.
    \item El juego tendra varios escenarios, donde habra que completar distintas misiones para ir al siguiente, cada uno mas dificil que el anterior o estara restringido por el nivel del personaje.
    \item Habra un sistema de nivel, del 1 al 10. Cuanto mas alto sea el nivel, mas puntos de vida tendra el jugador o algun otro beneficio , ya que tambien habran enemigos al cual enfrentar, estos nos impediran seguir adelante o trataran de obstaculizarnos de otras formas, pero al eliminarlos nos dara puntos de experiencia que nos permitiran subir de nivel.
    \item Al derrotar a los enemigos o completar tareas, se daran item que seran de ayuda para seguir adelante.
    \item En los escenarios habran cofres de tesoro, donde se podran encontrar distintos items como armamento, pociones o algun otro objeto interesante, aunque la aparicion de estos cofres del tesoro y la calidad de los item dependeran de la suerte y el nivel del jugador.
    \item El personaje que escoja el jugador tendra un estado, donde se mostrara los puntos de vida, puntos de ataque y de defensa, tambien se mostrara el nivel y los items que obtiene. El armamento y algunos items podran alterar las estadisticas del personaje.
\end{enumerate}
\subsection{Ideas Secundarias}

\begin{enumerate}

    \item Se manejara un sistema de credito, como monedas de oro, las cuale se podran utilizar para comprar items que nos ayudaran en el juego o podran venderlos para tener mas creditos para las siguientes aventuras.
    \item Habra un sistema de puntos, que demostrara su desempeño durante el progreso en los escenarios, por ahora estos puntos serviran como incentivo para que el jugador muestre un buen desempeño.
    \item Los enemigos a los que se enfrentaran seran de diversos tipos y tendran un rango, los cuales seran el comun, el raro y el Jefe. Los enemigos el ser derrotados aparte de dejar caer items, tambien dejaran caer dinero. Algunos enemigos raros dejaran caer cofres de tesoros pero los Jefes siempre dejaran caer un cofre del tesoro la primera vez que son derrotados, y estos cofres seran de buena calidad, dependiendo del nivel del personaje.
    \item El jugador solo podra selecionar un personaje masculino para el transcurso del juego.
    \item Los items tendran un sistema de rangos, que se basara en la calidad y beneficios que se obtendran al equiparlos, desde lo comun hasta lo legendario. Dependiendo de la calidad del item tambien dependera el valor monetario al momento de comparlos o venderlos.
\end{enumerate}
Por ahora estas seran las ideas en que se basara el desarrollo del juego, dependiendo de los contratiempos o  la productividas, se iran desallorando, implementando, ampliando o descartanto ideas en el trascurso del semestre. El logo del juego, las graficas y el desarrollo de los personajes como su imagen, tambien dependera de lo que se logre, ya sea segun las ideas establecidas o los contratiempos. 

Ademas, se debe recalcar que estas ideas se uniran con la de mi compañero, debido a que él ya tiene un inicio de la historia que se quiere plantear en el juego, por eso despues uniremos las ideas de ambos para que la historia pueda empezar a tomar un camino deseado.
\newpage
\section{Conclusion} \label{conclusion}
Para concluir, aunque no se tiene muchas ideas para el desarrollo de este proyecto, serviran como base para implementar y configurar el juego como mejor parezca, ya que estas ideas bases son de vital importantes para el tipo de juego que se quiere crear y daran bienvenidas a futuros cambios para mejorar, como el entretenimiento para los jugadores y la diversidad de escenas o para corregir fallos y llenar huecos que apareceran mas adelante.

\end{document}
